\documentclass[a4paper, 10pt]{article}
\usepackage[margin = 1in]{geometry}
\usepackage{amsmath}
\usepackage{tabularx}
\usepackage{framed}
\usepackage{array}
\usepackage{graphicx}
\usepackage{pdfpages}
\usepackage{fancyhdr}
\usepackage{titlesec}
\titleformat{\chapter}[hang]
{\normalfont\huge\bfseries}{\chaptertitlename\ \thechapter:}{1em}{}


\setlength{\parindent}{0em}
\newcolumntype{L}{>{\arraybackslash}m{10cm}}
\newcolumntype{T}{>{\arraybackslash}m{6cm}}

\begin{document}
\pagestyle{fancy}
\fancyhead[HR]{STAT-4330 \\ Homework 2 }
\fancyhead[HL]{Frank Ma}

1. Let $A$ be the matrix
\[
  A = \begin{bmatrix} 
     \dfrac{\strut 1}{\strut 2} & \dfrac{\strut 1}{\strut 2} \\
     \dfrac{\strut 1}{\strut 3} & \dfrac{\strut 2}{\strut 3}
  \end{bmatrix}
\] 

(a) Find the eigenvalues and eigenvectors for $A$. \\

To find the eigenvalues
\begin{align*}
   det(A - \lambda I) &= 0 \\
   det \left( 
      \begin{bmatrix} 
         \frac{1}{2} - \lambda & \frac{1}{2}   \\
         \frac{1}{3} & \frac{2}{3} - \lambda
      \end{bmatrix}
   \right) &= 0 \\
   \left( \frac{1}{2} - \lambda \right) \left( \frac{2}{3} - \lambda \right) - \frac{1}{6} &= 0 \\ 
   6 \lambda^2 - 7\lambda + 1 &= 0 \\
   (6 \lambda - 1)(\lambda - 1) &= 0 \\
\end{align*}

For $\lambda_1 = 1$ 
\begin{align*}
   \begin{bmatrix} 
      \frac{1}{2} & \frac{1}{2} \\
      \frac{1}{3} & \frac{2}{3} 
   \end{bmatrix} \begin{pmatrix} a \\b  \end{pmatrix}  = \begin{pmatrix} a \\b \end{pmatrix}  \\ 
      a &= b \\
      v_1 = k \begin{pmatrix} 1 \\ 1 \end{pmatrix}  \text{ for any }k
\end{align*}

For $\lambda_2 = \frac{1}{6}$ 
\begin{align*}
   \begin{bmatrix} 
      \frac{1}{2} & \frac{1}{2} \\
      \frac{1}{3} & \frac{2}{3} 
   \end{bmatrix} \begin{pmatrix} a \\b  \end{pmatrix}  = \frac{1}{6} \begin{pmatrix} a \\b \end{pmatrix}  \\ 
      \frac{1}{2} a + \frac{1}{2} b &= \frac{1}{6}a \\
      \frac{1}{3} a + \frac{2}{3} b &= \frac{1}{6} b \\
      b = \frac{-2}{3} a\\
      v_2 = k \begin{pmatrix} 1 \\ \frac{-2}{3} \end{pmatrix} \text{ for any $k$}
\end{align*}

(b) What is the determinant of $A$ \\

\begin{align*}
   det(A) = \frac{1}{3} - \frac{1}{6} = \frac{1}{6}
\end{align*}

(c) Let $D$ be a diagonalized matrix similar to $A$,
 \[
    D = \begin{bmatrix} 
       1 & 0 \\
       0 & \frac{1}{6}
    \end{bmatrix}
 \] 

\begin{align*}
   D &= P^{-1} A P \text{ where } P = \begin{bmatrix} 
      1 & 1 \\ 1 & - \frac{2}{3}
   \end{bmatrix} \\
     &= \left( - \frac{3}{5} \right) \begin{bmatrix} 
        - \frac{2}{3} & -1 \\ -1 & 1 
     \end{bmatrix} \begin{bmatrix} 
        \frac{1}{2} & \frac{1}{2} \\ \frac{1}{3}  & \frac{2}{3}
     \end{bmatrix} \begin{bmatrix} 
        1 & 1 \\ 1 & - \frac{2}{3}  
     \end{bmatrix} \\
     &= \left( - \frac{3}{5} \right)  \begin{bmatrix} 
        - \frac{5}{3} & 0 \\ 0 & - \frac{5}{18}  
     \end{bmatrix} \\
     &= \begin{bmatrix} 
        1 & 0 \\ 0 & \frac{1}{6}  
     \end{bmatrix}
\end{align*}

An expression for $A$ is
\[
   A = PDP^{-1}
\] 

An expression for $A^n$ is
 \begin{align*}
    A^n &= \left( PDP^{-1} \right)^n \\
        &= PDP^{-1} \cdot PDP^{-1} \hdots PDP^{-1} \\
        &= PD^n P^{-1} \\
        &= \begin{bmatrix} 
           1 & 1 \\ 1 & - \frac{2}{3}
        \end{bmatrix} \begin{bmatrix} 
           1 & 0 \\ 0 & \left( \frac{1}{6} \right)^n
        \end{bmatrix} \begin{bmatrix} 
           - \frac{2}{3}   & -1 \\ -1 & 1
        \end{bmatrix} \left( \frac{-3}{5} \right) 
\end{align*}

(d) Trace of $A$ is $ \frac{1}{2} + \frac{2}{3} = \frac{7}{6} $

(e) The transpose of $A$ is 

\[
  A^T = \begin{bmatrix} 
   \frac{\strut 1}{\strut 2} & \frac{\strut 1}{\strut 3}   \\ 
   \frac{\strut 1}{\strut 2} & \frac{\strut 2}{\strut 3}
\end{bmatrix}
\] 

The eigenvalues of $A^T$ are the roots to

\begin{align*}
   det \left( A^T - \lambda I \right)  &= 0 \\
   \left| \begin{bmatrix} 
      \frac{1}{2} - \lambda & \frac{1}{3} \\ \frac{1}{2} & \frac{2}{3} - \lambda  
   \end{bmatrix}
   \right|  &= 0 \\
   \left( \frac{1}{2} - \lambda \right) \left( \frac{2}{3} - \lambda \right) - \frac{1}{6} &= 0 \\
\end{align*}

The eigenvalues are $\lambda_1 = 1$ and $\lambda_2 = \frac{1}{6}$. \\

For $\lambda_1 = 1$, 

\begin{align*}
   \begin{bmatrix} 
      \frac{1}{2} & \frac{1}{3} \\ \frac{1}{2} & \frac{2}{3}    
      \end{bmatrix} \begin{pmatrix} a \\b \end{pmatrix}  &= \begin{pmatrix} a \\ b \end{pmatrix} 
\end{align*}

hence
\begin{align*}
      \frac{1}{2} a + \frac{1}{3}b = a \\
      \frac{1}{2} a + \frac{2}{3} b = b \\
      b &= \frac{3}{4}a 
\end{align*}

Therefore $v_1 = k \begin{pmatrix} 1 \\ \frac{3}{4} \end{pmatrix} $ for any $k$ \\

For  $\lambda_2 = \frac{1}{6}$
\begin{align*}
   \begin{bmatrix} 
      \frac{1}{2} & \frac{1}{3} \\ \frac{1}{2} & \frac{2}{3}    
      \end{bmatrix} \begin{pmatrix} a \\b \end{pmatrix}  &= \frac{1}{6}\begin{pmatrix} a \\ b \end{pmatrix} 
\end{align*}

hence
\begin{align*}
      \frac{1}{2} a + \frac{1}{3}b = \frac{1}{6}a \\
      \frac{1}{2} a + \frac{2}{3} b = \frac{1}{6}b \\
      b &= -a
\end{align*}

Therefore $v_2 = k \begin{pmatrix} 1 \\ -1 \end{pmatrix} $ for any $k$ \\













\end{document}

