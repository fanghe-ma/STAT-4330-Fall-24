\chapter{}

\today \\

Tossing a fair coin 100 times, $X$ the number of heads is a Binomial R.V. \\

\textbf{$X = 50$}

\[
   P(X = 50) = \begin{pmatrix} 100 \\ 50 \end{pmatrix} \left( \frac{1}{2} \right)^{100} = \frac{100!}{50! 50!} \frac{1}{2}^{100}
\] 

Factorials can be estimated with Stirling's Formula

\[
   n \sim \sqrt{2\pi} n^{n + \frac{1}{2}} e^{-n}
\] 

$\sim$ implies
\[
   \lim_{n \to \infty} \frac{n!}{\sqrt{2\pi} n^{n + \frac{1}{2}} e^{-n}} = 1
\] 

i.e. The absolute error might be large (tends to infinity actually) but 'relative error' is small \\

\textbf{$45 \leq X \leq 55$} \\

To find $P(45 \leq X \leq 55)$ requires summation over Binomial pdf. Or, $X$ is approximately normal by the central limit theorem. \\

 This can also be done for $P(X = 50)$, by evaluating the density at $x= 50$ in a normal pdf, or integrating from $49.5$ to $50.5$ maybe.  \\

\textbf{$X = 70?$} \\

Is the normal approximation still good?\\

\textbf{Proof of CLT} \\

'Proofs' for CLT likely proceeds with moment generating functions but glosses over the hard part of the proof.  \\

\textbf{Moving to a stochastic process} \\

Imagine plotting $S_n$ against $n$, where $S_n$ is the number of heads - number of tails up to 'time' $n$.  \\

Now the question of $P(X = 50)$ is asking for the probability that $S_n = 0$ when $n = 100$. \\

Possible questions
\begin{itemize}
    \item what is the chance $S_n$ always stays above 0?
    \item if $n$ goes to infinity, will $S_n$ always hit 0 at some point (yes btw) ? Is there a chance it doesn't hit 0?
    \item if coin is not fair ( $p > \frac{1}{2}$), what will happen to $S_n$ as $n \rightarrow \infty$
        \begin{itemize}
           \item $S_n$ will 'drift' above the axis, $n$ will be above the $x$ axis (proof via \textbf{strong law of large numbers})
       \end{itemize}
    \item if $S_n$ returns to $0$, what do we know about $n$ when it returns (other than $n$ is even)
    \item is the expected time to return less than 10, between 10 to 20, or 20 to 100? 
    \item let $Y$ be the R.V. denoting the last $n$ at which $S_n$ touched the $x$ axis. What kind of RV is this?
       \begin{itemize}
          \item it is not obvious, but $Y$ is a symmetrical RV about $50$ 
          \item is $Y = 50$ more likely than $Y = 0$ (i.e. $Y = 100$)? 
       \end{itemize}
    \item for 100 tries, is it more likely that the last time $S_n = 0$  occurs at $n = 2$ or $n = 98$ (they are equally likely)
    \item what fraction of the time is $S_n$ above the axis? What fraction is it below? 
    \item what is $P(S_1 > 0, S_2 > 0, S_3 > 0 \hdots S_2n = r)$ OR, what is the probability that $S_n$ takes some path above the $x$ axis and ends at $(2n, r)$
    \begin{itemize}
       \item any path of length $2n$ has the same probability  $ \frac{1}{2}^{2n}$ 
       \item any path that ends at a specified spot as a binomial probability
       \item to find the above probability, we count the number of paths that satisfy the condition
    \end{itemize}
    
\end{itemize}

